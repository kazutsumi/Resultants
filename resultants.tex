\documentclass[12pt, uplatex, dvipdfmx]{jsarticle} 
\usepackage{amsmath,amsthm,amssymb,amsfonts,fancyhdr, enumerate, braket, setspace, graphicx, bm, multirow, bigdelim}

\newcommand{\ds}{\displaystyle}

\theoremstyle{definition}
\newtheorem{theorem}{定理}
\newtheorem*{theorem*}{定理}
\newtheorem*{definition}{定義}
\newtheorem{example}{例}
\newtheorem*{remark}{注意}
\newtheorem*{promis}{約束}
\renewcommand{\proofname}{\textbf{証明}}
\renewcommand{\thesection}{\arabic{section}}

\DeclareMathOperator{\Syl}{Syl}
\DeclareMathOperator{\resul}{resul}
\DeclareMathOperator{\Rat}{Rat}

\pagestyle{plain}


\begin{document}

\section{1変数多項式の終結式}

多項式の係数はいずれも整域 $R$ の元としておく.こうしておくと多項式の係
数を商体 $\Rat(R)$ の元として分数に拡張できる.さらに方程式の解は代数閉
包 $\overline{\Rat(R)}$ 上で考えられる.もしかしたら $R$ を UFD くらい
に仮定しといた方が安全かもしれない.


\begin{definition}[\textbf{終結式}]
  多項式 $\ds f(x) = \sum_{i=0}^{m}a_m x^i$ と $\ds g(x) = \sum_{j=0}^{n} b_n x^j \; (a_m, b_n \neq 0)$ に対し
  て
  \[
    \Syl(f,g):=
    \begin{array}{rccccccccll}
      \ldelim[{9}{12pt}[] &  a_m & a_{m-1} & \cdots & a_1 & a_0 & & & & \rdelim]{9}{12pt}[] & \rdelim\}{4}{14pt}[\(n\)]\\
                          &  & a_m & a_{m-1} & \cdots & a_1 & a_0& & & &\\
                          &  & & \ddots & \ddots & & \ddots & \ddots & & & \\
                          &  & & & a_m & a_{m-1} & \cdots & a_1 & a_0 & &\\
                          & b_n & b_{n-1} & \cdots & b_1 & b_0 & & & & &  \rdelim\}{4}{14pt}[\(m\)]\\
                          & & b_n & b_{n-1} & \cdots & b_1 & b_0 & & & & \\
                          & & & \ddots & \ddots & & \ddots & \ddots & & &\\
                          & & & & b_n & b_{n-1} & \cdots & b_1 & b_0 & &
    \end{array}
  \]
  を $f$ と $g$ の\textbf{シルベスター行列}といい,その行列式
  \[
    \resul(f,g):= \det(\Syl(f,g))
  \]
  を $f$ と $g$ の\textbf{終結式 (resultant)}という.なお,零でない定数 $g(x) = b_0 \neq 0$ に対しては
  \[
    \Syl(f, b_0) =\left[
    \begin{array}{ccc}
       b_0  & & \\
            & \ddots & \\
            & & b_0 
    \end{array}
  \right], \quad \resul(f, b_0) = b_0^m
  \]
  と定める.同様に,$\resul(a_0, g) = a_m^n$ とする.さらに,零多項式に対して
  は $\resul(f,0) = \resul(0,g)=0$ と定める.
\end{definition}

\begin{remark}
  上で定義した $\Syl(f,g)$ の転置行列をシルベスター行列と呼ぶ流儀もある.
\end{remark}

\begin{example}
  $f(x) = x^3+1, \; g(x)=x^2+2x+1, \; h(x) = x^2+1$ とする.
  \[
    \resul(f,g) = \left| 
      \begin{array}{ccccc}
        1 & 0 & 0 & 1 & 0\\
        0 & 1 & 0 & 0 & 1\\
        1 & 2 & 1 & 0 & 0\\
        0 & 1 & 2 & 1 & 0\\
        0 & 0 & 1 & 2 & 1
      \end{array}
    \right|=2, \quad \resul(f,h) = \left|
      \begin{array}{ccccc}
        1 & 0 & 0 & 1 & 0\\
        0 & 1 & 0 & 0 & 1\\
        1 & 0 & 1 & 0 & 0\\
        0 & 1 & 0 & 1 & 0\\
        0 & 0 & 1 & 0 & 1
      \end{array}
    \right|=0
  \]
\end{example}

終結式に関するいくつかの性質を紹介する.主に線形代数との関わりを紹介し
たいので,主張や証明が重複することが多々ある.以下,$f,g$ は次
のような多項式とする.
\[
  f(x)=\sum_{i=0}^{m} a_i x^i, \; g(x)=\sum_{j=0}^{n} b_j x^j
  , \quad a_m, b_n \neq 0
\]

\begin{theorem}
  $\resul(f,g) = 0$ ならば,$f$ と $g$ は共通根を持つ.
\end{theorem}

\begin{proof}
  $f(\alpha)=g(\alpha)=0$ とする.
  \[
    \begin{aligned}
      \alpha^i f(\alpha) &= a_m \alpha^{m+i} + a_{m-1}\alpha^{m-1+i} +
      \cdots + a_0 \alpha^i =0 \quad (i=0,1, \ldots, n-1),\\
      \alpha^j g(\alpha) &= b_n \alpha^{n+j} + b_{n-1}\alpha^{n-1+j} +
      \cdots + b_0 \alpha^j=0 \quad (j=0, 1, \ldots, m-1)
    \end{aligned}
  \]
  が成り立つ.これは行列を使って以下のように書ける.
  \[
    \Syl(f,g) \left[
      \begin{array}{c}
        \alpha^{m+n-1}\\
        \alpha^{m+m-2}\\
        \vdots\\
        \alpha\\
        1
      \end{array}
    \right] = \bm{0}_{m+n}
  \]
  同次形連立1次方程式 $\Syl(f,g) \bm{x} = \bm{0}$ が
  非自明解を持つので $\resul(f,g)=0$ である.
\end{proof}

\begin{theorem}
  定数でない多項式 $f,g$ に関して以下は同値である.
  \begin{enumerate}[(1)]
  \item $\resul(f,g) = 0$.
  \item $f, g$ は定数でない共通因子を持つ.%($\Leftrightarrow f, g$ は共通根を持つ. )
  \item 以下を満たす多項式 $A, B$ (少なくとも一方は非零多項式)が存在する.
    \[
      Af + Bg = 0, \quad \deg A <n, \; \deg B <m
    \]
  \end{enumerate}
\end{theorem}

\begin{proof}
  
\end{proof}

\newpage

\section{2変数多項式の終結式}


\begin{thebibliography}{99}
\bibitem{NI} 長坂工作・岩根秀直(編),『計算機代数の基礎理論』,共立出版 (2020).

\bibitem{miyake} 三宅敏恒,『線形代数概論』,培風館 (2023).

\bibitem{yokoyama} 横山和弘,『多項式と計算機代数』,朝倉書店 (2022).

\bibitem{CLO} D. A. Cox, J. Little and D. O'Shea, \textit{Ideals Varieties, and Algorithms 4th edition}, Springer (2015).

\bibitem{Lang} S. Lang, \textit{Algebra Revised 3rd edition}, Springer (2004). 
\end{thebibliography}

\end{document}
