\documentclass[12pt, uplatex, dvipdfmx]{jsarticle} 
\usepackage{amsmath,amsthm,amssymb,amsfonts,fancyhdr, enumerate, braket, setspace, graphicx, bm, multirow, bigdelim}

\newcommand{\ds}{\displaystyle}

\theoremstyle{definition}
\newtheorem{theorem}{定理}
\newtheorem*{theorem*}{定理}
\newtheorem*{definition}{定義}
\newtheorem{example}{例}
\newtheorem*{remark}{注意}
\newtheorem*{promis}{約束}
\renewcommand{\proofname}{\textbf{証明}}
\renewcommand{\thesection}{\arabic{section}}

\DeclareMathOperator{\Syl}{Syl}
\DeclareMathOperator{\resul}{resul}
\DeclareMathOperator{\Rat}{Rat}

\pagestyle{fancy}

\chead{終結式}

\begin{document}

\section*{1変数多項式の終結式}

多項式の係数はいずれも整域 $R$ の元としておく.こうしておくと多項式の係
数を商体 $\Rat(R)$ の元として分数に拡張できる.さらに方程式の解は代数閉
包 $\overline{\Rat(R)}$ 上で考えられる.もしかしたら $R$ を UFD くらい
に仮定しといた方が安全かもしれない.


\begin{definition}[\textbf{終結式}]
  多項式 $\ds f(x) = \sum_{i=0}^{m}a_m x^i$ と $\ds g(x) = \sum_{j=0}^{n} b_n x^j \; (a_m, b_n \neq 0)$ に対し
  て
  \[
    \Syl(f,g):=
    \begin{array}{rccccccccll}
      \ldelim[{9}{12pt}[] &  a_m & a_{m-1} & \cdots & a_1 & a_0 & & & & \rdelim]{9}{12pt}[] & \rdelim\}{4}{14pt}[\(n\)]\\
                          &  & a_m & a_{m-1} & \cdots & a_1 & a_0& & & &\\
                          &  & & \ddots & \ddots & & \ddots & \ddots & & & \\
                          &  & & & a_m & a_{m-1} & \cdots & a_1 & a_0 & &\\
                          & b_n & b_{n-1} & \cdots & b_1 & b_0 & & & & &  \rdelim\}{4}{14pt}[\(m\)]\\
                          & & b_n & b_{n-1} & \cdots & b_1 & b_0 & & & & \\
                          & & & \ddots & \ddots & & \ddots & \ddots & & &\\
                          & & & & b_n & b_{n-1} & \cdots & b_1 & b_0 & &
    \end{array}
  \]
  を $f$ と $g$ の\textbf{シルベスター行列}といい,その行列式
  \[
    \resul(f,g):= \det(\Syl(f,g))
  \]
  を $f$ と $g$ の\textbf{終結式 (resultant)}という.なお,零でない定数 $g(x) = b_0 \neq 0$ に対しては
  \[
    \Syl(f, b_0) =\left[
    \begin{array}{ccc}
       b_0  & & \\
            & \ddots & \\
            & & b_0 
    \end{array}
  \right], \quad \resul(f, b_0) = b_0^m
  \]
  と定める.同様に,$\resul(a_0, g) = a_m^n$ とする.さらに,零多項式に対して
  は $\resul(f,0) = \resul(0,g)=0$ と定める.
\end{definition}

\begin{example}
  $f(x) = x^3+1, \; g(x)=x^2+2x+1, \; h(x) = x^2+1$ とする.
  \[
    \resul(f,g) = \left| 
      \begin{array}{ccccc}
        1 & 0 & 0 & 1 & 0\\
        0 & 1 & 0 & 0 & 1\\
        1 & 2 & 1 & 0 & 0\\
        0 & 1 & 2 & 1 & 0\\
        0 & 0 & 1 & 2 & 1
      \end{array}
    \right|=2, \quad \resul(f,h) = \left|
      \begin{array}{ccccc}
        1 & 0 & 0 & 1 & 0\\
        0 & 1 & 0 & 0 & 1\\
        1 & 0 & 1 & 0 & 0\\
        0 & 1 & 0 & 1 & 0\\
        0 & 0 & 1 & 0 & 1
      \end{array}
    \right|=0
  \]
\end{example}


\newpage

\section*{2変数多項式の終結式}

\end{document}
