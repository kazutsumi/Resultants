\documentclass[12pt, uplatex, dvipdfmx]{jsarticle} 
\usepackage{amsmath,amsthm,amssymb,amsfonts,fancyhdr, enumerate, braket, setspace, graphicx, bm, multirow, bigdelim}
\newcommand{\ds}{\displaystyle}

\theoremstyle{definition}
\newtheorem{theorem}{定理}
\newtheorem*{theorem*}{定理}
\newtheorem*{definition}{定義}
\newtheorem{lemma}{補題}
\newtheorem{example}{例}
\newtheorem*{remark}{注意}
\newtheorem*{promis}{約束}
\renewcommand{\proofname}{\textbf{証明}}
\renewcommand{\thesection}{\arabic{section}}

\DeclareMathOperator{\Syl}{Syl}
\DeclareMathOperator{\resul}{resul}
\DeclareMathOperator{\Rat}{Rat}
\DeclareMathOperator{\disc}{disc}

\pagestyle{plain}


\begin{document}

\section{1変数多項式の終結式}

多項式の係数はいずれも整域 $R$ の元としておく.こうしておくと多項式の係
数を商体 $\Rat(R)$ の元として分数に拡張できる.さらに方程式の解は代数閉
包 $\overline{\Rat(R)}$ 上で考えられる.もしかしたら $R$ を UFD くらい
に仮定しといた方が安全かもしれない.


\begin{definition}[\textbf{終結式}]
  多項式 $\ds f(x) = \sum_{i=0}^{m}a_m x^i$ と $\ds g(x) = \sum_{j=0}^{n} b_n x^j \; (a_m, b_n \neq 0)$ に対し
  て
  \[
    \Syl(f,g):=
    \begin{array}{rccccccccll}
      \ldelim[{9}{12pt}[] &  a_m & a_{m-1} & \cdots & a_1 & a_0 & & & & \rdelim]{9}{12pt}[] & \rdelim\}{4}{14pt}[\(n\)]\\
                          &  & a_m & a_{m-1} & \cdots & a_1 & a_0& & & &\\
                          &  & & \ddots & \ddots & & \ddots & \ddots & & & \\
                          &  & & & a_m & a_{m-1} & \cdots & a_1 & a_0 & &\\
                          & b_n & b_{n-1} & \cdots & b_1 & b_0 & & & & &  \rdelim\}{4}{14pt}[\(m\)]\\
                          & & b_n & b_{n-1} & \cdots & b_1 & b_0 & & & & \\
                          & & & \ddots & \ddots & & \ddots & \ddots & & &\\
                          & & & & b_n & b_{n-1} & \cdots & b_1 & b_0 & &
    \end{array}
  \]
  を $f$ と $g$ の\textbf{シルベスター行列}といい,その行列式
  \[
    \resul(f,g):= \det(\Syl(f,g))
  \]
  を $f$ と $g$ の\textbf{終結式 (resultant)}という.なお,零でない定数 $g(x) = b_0 \neq 0$ に対しては
  \[
    \Syl(f, b_0) =\left[
    \begin{array}{ccc}
       b_0  & & \\
            & \ddots & \\
            & & b_0 
    \end{array}
  \right], \quad \resul(f, b_0) = b_0^m
  \]
  であり,同様に $\resul(a_0, g) = a_m^n$ である.また,共に定数の場合
  は $\resul(a_0, b_0)=1$ とし,一方が零多項式の場合は $\resul(f,0) =
  \resul(0,g)=0$ と定める.
\end{definition}

\begin{remark}
  上で定義した $\Syl(f,g)$ の転置行列をシルベスター行列と呼ぶ流儀もある.
\end{remark}

\begin{example}\label{exmp:fgh}
  $f(x) = x^3+1, \; g(x)=x^2+2x+1, \; h(x) = x^2+1$ とする.
  \[
    \resul(f,g) = \left| 
      \begin{array}{ccccc}
        1 & 0 & 0 & 1 & 0\\
        0 & 1 & 0 & 0 & 1\\
        1 & 2 & 1 & 0 & 0\\
        0 & 1 & 2 & 1 & 0\\
        0 & 0 & 1 & 2 & 1
      \end{array}
    \right|=0, \quad \resul(f,h) = \left|
      \begin{array}{ccccc}
        1 & 0 & 0 & 1 & 0\\
        0 & 1 & 0 & 0 & 1\\
        1 & 0 & 1 & 0 & 0\\
        0 & 1 & 0 & 1 & 0\\
        0 & 0 & 1 & 0 & 1
      \end{array}
    \right|=2
  \]
  次の定理\ref{thm:common-solution}から,$f,g$ は共通根を持ち,$f,h$ は共通根を持たないことがわかる.
\end{example}

以下,$f,g$ は次のような多項式とする.ただし,$a_m, b_n \neq 0$ とする.
\[
  f(x)=\sum_{i=0}^{m} a_i x^i = a_m\prod_{i=1}^{m}(x-\alpha_i), \quad
  g(x)=\sum_{j=0}^{n} b_j x^j=b_n\prod_{j=1}^{n}(x-\beta_j)
\]

\begin{theorem}\label{thm:common-solution}
$f$ と $g$ は共通根を持つ.$\Longleftrightarrow$ \;$\resul(f,g)=0$.
\end{theorem}

\begin{proof}
  $(\Rightarrow)$ $f(\gamma)=g(\gamma)=0$
  とすると,$\gamma^i f(\gamma) = \gamma^j g(\gamma)=0$ なので以下が成
  り立つ.
  \[
    \left[
      \begin{array}{ccccc}
        a_3 & a_{2} & a_1 & a_0 & 0\\
        0 & a_3 & a_2 & a_1 & a_0\\
        b_2 & b_1 & b_0 & 0 & 0\\
        0 & b_2 & b_1 & b_0 & 0\\
        0 & 0 & b_2 & b_1 & b_0
      \end{array}
    \right]\left[
      \begin{array}{c}
        \gamma^4\\
        \gamma^3\\
        \gamma^2\\
        \gamma\\
        1
      \end{array}
    \right] = \left[
      \begin{array}{c}
        0\\
        0\\
        0\\
        0\\
        0
      \end{array}
    \right] \quad (m=3, n=2 \text{ の場合})
  \]
  同次形連立1次方程式 $\Syl(f,g) \bm{x} = \bm{0}$ が
  非自明解を持つので $\resul(f,g)=0$ である.

  $(\Leftarrow)$ 以下の補題\ref{lem:explicit-resul}より従う.
\end{proof}

\begin{lemma}\label{lem:explicit-resul}
  $\ds \resul(f,g) =  a_m^n b_n^m \prod_{i=1}^{m} \prod_{j=1}^{n} (\alpha_i-\beta_j)$
\end{lemma}

\begin{proof}
  定理\ref{thm:common-solution}の $(\Rightarrow)$ から $\alpha_i=\beta_j$ のと
  き $\resul(f,g)=0$ なので,因数定理より $T:=\prod \prod (\alpha_i-\beta_j)$ は $\resul(f,g)$ を割り切
  る.そこで,各 $\alpha_i, \beta_j$ の多項式とし
  て $\resul(f,g)$ と $T$ の次数を評価し,係数を比較すればよい.
  \[
    \begin{aligned}
      f(x)/a_m &= x^m + A_{1} x^{m-1} + \cdots + A_m = \prod_{i=1}^{m}(x-\alpha_i),\\
      g(x)/b_n &= x^n + B_{1} x^{n-1} + \cdots + B_n = \prod_{j=1}^{n}(x-\beta_j)
    \end{aligned}
  \]
  とすると,$m=3, n=2$ の場合
  \[
    \resul(f,g) = a_3^2\ b_2^3\ \left|
      \begin{array}{ccccc}
        1 & A_1 & A_2  & A_3 & 0\\
        0 & 1 & A_1 & A_2 & A_3\\
        1 & B_1 & B_2 & 0 & 0 \\
        0 & 1 & B_1 & B_2 & 0\\
        0 & 0 & 1 & B_1 & B_2
      \end{array}
    \right|
  \]
  である.各 $A_k$ は $\alpha_1, \ldots, \alpha_m$ の,
  各 $B_k$ は $\beta_1, \ldots, \beta_n$ の $k$ 次基本対称式の $\pm 1$
  倍である.また,各 $A_i, B_j$ はそれぞれ $\alpha_i, \beta_j$ の1次式
  なので,$\resul(f,g)$ は $\alpha_i$ に関して高々 $n$ 次で,$\beta_j$
  に関して高々 $m$ 次である.従って,$\resul(f,g)$ は $T$ の定数倍であ
  り,$\alpha_1^n$ の係数を比較して,$\resul(f,g) = a_m^n b_n^m T$ がわ
  かる.
\end{proof}

\begin{example}\label{exmp:disc2}
  $f(x) = ax^2+bx+c \; (a \neq 0)$ とする.$f'(x) = 2ax + b$ であり,
  \[
    \resul(f, f') = \left|
      \begin{array}{ccc}
        a & b & c\\
        2a & b & 0\\
        0 & 2a & b
      \end{array}
    \right| =-a (b^2-4ac)
  \]
  より,$b^2-4ac=0$ のとき $f, f'$ は共通根を持つ.また,このとき $f$ は重根を持つ.
\end{example}

\begin{theorem}\label{thm:double-resul}
  非定数多項式 $f$ に対して,以下は同値.
  \begin{enumerate}[(1)]
  \item $f$ は重根を持つ.
  \item $f, f'$ は共通根を持つ.
  \item $\resul(f, f')=0$.
  \end{enumerate}
\end{theorem}

% \begin{proof}
%   $(1) \Leftarrow (2)$ $\alpha$ を $f$ の重根とすると,$f(x) = (x-\alpha)^2q(x)$ と書ける.このとき
%   \[
%     f'(x) = 2(x-\alpha)q(x) + (x-\alpha)^2 q'(x) = (x-\alpha)\left( 2q(x) + (x-\alpha)q'(x)\right)
%   \]
%   より,$f, f'$ は共通根 $\alpha$ を持つ.

%   $(2) \Rightarrow (1)$ $\alpha$ を $f(\alpha)=f'(\alpha)=0$ とす
%   る.$f(x)=(x-\alpha)p(x)$ と書ける.
%   \[
%     f'(x) = p(x) + (x-\alpha)p'(x)
%   \]
%   より $f'(\alpha) = p(\alpha) =0$ から $p(x) =(x-\alpha)q(x)$ と書けるので, $\alpha$ は $f$ の重根である.

%   $(2) \Leftrightarrow (3)$ 定理\ref{thm:common-solution}より従う.
% \end{proof}

\begin{theorem}\label{thm:resul-disc}
  \[
    \resul(f,f') = (-1)^{m(m-1)/2}\ a_m^{2m-1} \prod_{1 \leq i < j \leq m} (\alpha_i - \alpha_j)^2
  \]
\end{theorem}

\begin{proof}
  補題\ref{lem:explicit-resul}より,一般に
  \[
    \resul(f,g) = a_m^n \ b_n^m\ \prod_{i=1}^{m} \prod_{j=1}^{n} (\alpha_i-\beta_j)
    =a_m^n \ \prod_{i=1}^{m} \left( b_n \prod_{j=1}^{n}(\alpha_i-\beta_j)\right) = a_m^n\ \prod_{i=1}^{m} g(\alpha_i)
  \]
  が成り立つ.これを $g=f'$ として適用して
  \[
    \resul(f, f') = a_m^{m-1} \ \prod_{i=1}^{m} f'(\alpha_i)
  \]
  を得る.$\ds f'(x) = a_m\ \sum_{i=1}^{m} \prod_{\substack{j=1 \\ j \neq
      i}}^{m} (x-\alpha_j)$ より
  $\ds f'(\alpha_i) = a_m \prod_{\substack{j=1 \\ j \neq i}}
  (\alpha_i-\alpha_j)$ から従う.
\end{proof}

\begin{definition}[\textbf{判別式}] $2$ 次以上の多項式 $f$ に対して以下の $\disc(f)$ を $f$ の\textbf{判別式}という.
  \[
    \disc(f) = a_m^{2m-2} \prod_{1 \leq i < j \leq m} (\alpha_i-\alpha_j)^2 = \frac{(-1)^{m(m-1)/2}}{a_m} \resul(f,f')
  \]
  定理\ref{thm:double-resul}, \ref{thm:resul-disc}より,$f$ が重根を持つことと $\disc(f)=0$ は同値である.
\end{definition}

\begin{example}
  $f(x) = x^3+px+q$ とする.$f'(x) = 3x^2+p$ より, $\disc(f)$ は以下の通り.
  \[
    \disc(f) = (-1)^3 \resul(f,f') = -\left|
      \begin{array}{ccccc}
        1 & 0 & p & q & 0\\
        0 & 1 & 0 & p & q\\
        3 & 0 & p & 0 & 0\\
        0 & 3 & 0 & p & 0\\
        0 & 0 & 3 & 0 & p
      \end{array}
    \right| = -(4p^3+27q^2)
  \]
\end{example}

\newpage

\begin{theorem}
  $f$ を実係数 $3$ 次多項式とする.
  \[
    \disc (f) =
    \begin{cases}
      >0 & (\text{ $f$ は相異なる3個の実根を持つ})\\
      =0 & (\text{ $f$ は重根を持ち,どの根も実数})\\
      <0 & (\text{ $f$ は1個の実根と2個の互いに共役な虚根を持つ})
    \end{cases}
  \]
\end{theorem}

\begin{proof}
  $f$ の根 $\alpha_1, \alpha_2, \alpha_3$ が互いに相異なる実数のとき,
  定義から$\disc(f) >0$ である.$f$ が重根を持つとき,$\disc(f)=0$ であ
  る.$\alpha_1$ が実数で $\alpha_2, \alpha_3$ が互いに共役な虚数のとき,
  \[
    \begin{aligned}
      \disc(f) &= a_m^{2m-2} (\alpha_1-\alpha_2)^2(\alpha_1-\alpha_3)^2(\alpha_2-\alpha_3)^2\\
      &=a_m^{2m-2}\left( \left( \alpha_1 - \alpha_2\right)\left(\overline{\alpha_1-\alpha_2}\right)\right)^2
      \left( 2 i \left(\Im \alpha_2\right)\right)^2
      = -4 a_m^{2m-2} \left| \alpha_1-\alpha_2\right|^2 \left( \Im \alpha_2\right)^2 <0
    \end{aligned}
  \]
  である.$f$ は実係数なので実根は1個以上あり,虚根は偶数個で重根ではない.
\end{proof}


\begin{example}[接点 $t$ 問題] 点 $(a,b)$ から曲線 $y=x^3-3x$ に引ける
  接線の本数が $3$ 本になるときの $a,b$ の条件を求めよう.\\

  点 $(a,b)$ を通る直線 $y=m(x-a)+b$ と曲線 $y=x^3-3x$ が接するための必
  要十分条件は,$3$ 次多項式 $f(x) = x^3-3x - \left( m(x-a)+b\right)$
  が重根を持つこと,つまり
  \[
    \resul(f, f') = -4 m^{3}+9 \left(3 a^{2}-4\right) m^{2}-54 \left(a
      b +2\right) m +27 \left(b -2\right) \left(b +2\right)=0
  \]
  が成り立つことである.そして,このような接線が $3$ 本存在することは,
  上の $m$ に関する $3$ 次方程式が異なる $3$ 実解を持つことと同値であ
  る.つまり,
  \[
    g(m) = -4 m^{3}+9 \left(3 a^{2}-4\right) m^{2}-54 \left(a
      b +2\right) m +27 \left(b -2\right) \left(b +2\right)
  \]
  とおいて,$\disc(g)>0$ となる条件を求めればよい.
  \[
    \disc(g) = \frac{-1}{-4}\resul(g,g') = 314928 (a^3-3a-b)(3a+b)^3
  \]
  より,$(a^3-3a-b)(3a+b)>0$ が求める条件である.
\end{example}

\newpage

\begin{theorem}\label{thm:common-factor}
  定数でない多項式 $f,g$ に関して以下は同値である.
  \begin{enumerate}[(1)]
  \item $f, g$ は定数でない共通因子を持つ.%($\Leftrightarrow f, g$ は共通根を持つ. )
  \item 以下を満たす多項式 $U, V$ (少なくとも一方は非零多項式)が存在する.
    \[
      Uf + Vg = 0, \quad \deg U <n, \; \deg V <m
    \]
  \item $\resul(f,g) = 0$.
  \end{enumerate}
\end{theorem}

\begin{proof}
  $(1) \Rightarrow (2)$ $h$ を $f,g$ の共通因子とし,$f=hf_1, \,
  g=hg_1$ とする.
  \[
    g_1 \cdot f + (-f_1) \cdot g = g_1  hf_1 - f_1 hg = 0
  \]
  より,$U=g_1, V=-f_1$ とすればよい.

  $(2) \Rightarrow (1)$
  $Uf+Vg=0, \; \deg U<n, \, \deg V<m, \; V \neq 0$ とする.$f,g$ が共通
  因子を持たないとすると,$\tilde{U}f+\tilde{V}g=1$ を満たす多項
  式 $\tilde{U}, \tilde{V}$ が存在する.$Vg = -Uf$ なので
  \[
    V = V(\tilde{U}f+\tilde{V}g) = \tilde{U}Vf + \tilde{V}Vg = \tilde{U}Vf + \tilde{V}(-Uf) = (\tilde{U}V - \tilde{V}U)f
  \]
  である.$V \neq 0$ なので $\deg V \geq \deg f = n$ より,$\deg V <n$ に矛盾する.

  $(2) \Leftrightarrow (3)$ 簡単のため,$m=3, n=2$
  とする.一般の場合も同様である.
  \[
    U= \sum_{i=0}^{n-1} u_i x^i=u_1 x + u_0, \quad V= \sum_{j=0}^{m-1} v_j x^j=v_2 x^2+v_1 x + v_0
  \]
  とおき,$\bm{w} = \left[
    \begin{array}{ccccc}
      u_1 & u_0 & v_2 & v_1 & v_0
    \end{array}
    \right]^{\top}$ とすると
  \[
    \begin{aligned}
      Uf+Vg=0 &\Leftrightarrow
      \begin{cases}
        a_3u_1 + b_2v_2=0\\
        a_2u_1+a_3u_0+b_1v_2+b_2v_1=0\\
        a_1u_1+a_2u_0+b_0v_2+v_1v_1+b_2v_0=0\\
        a_0u_1+a_1u_0+b_0v_1+b_1v_0=0\\
        a_0u_0+b_0v_0=0
      \end{cases}\\
      & \Leftrightarrow \left[
        \begin{array}{ccccc}
          a_3 & 0 & b_2 & 0 & 0\\
          a_2 & a_3 & b_1 & b_2 & 0\\
          a_1 & a_2 & b_0 & b_1 & b_2\\
          a_0 & a_1 & 0 & b_0 & b_1\\
          0 & a_0 & 0 & 0 & b_0
        \end{array}
      \right] \left[
        \begin{array}{c}
          u_1\\
          u_0\\
          v_2\\
          v_1\\
          v_0
        \end{array}
      \right] = \bm{0}_5
      \Leftrightarrow \Syl(f,g)^{\top} ~\bm{w}= \bm{0}_5
    \end{aligned}
  \]
  なので,$\resul(f,g) = \det\left(\Syl(f,g)^{\top}\right)$ と合わせて以下を得る.
  \[
    (2) \Leftrightarrow \text{同次形連立1次方程式 } \Syl(f,g)^{\top}~\bm{x}=\bm{0} \text{ が非自明解を持つ}
    \Leftrightarrow (3)
  \]
\end{proof}


\begin{theorem}\label{thm:gcdlike}
  非零多項式 $f,g \in R[x]$ に対して次を満たす $U,V \in \Rat(R)[x]$ が存在する.
  \[
    Uf + V g = \resul(f,g)
  \]
  特に,$f,g$ の一方が非定数なら,$U, V \in R[x]$ である.
\end{theorem}

\begin{proof}
  $\resul(f,g)=0$ なら $U=V=0$ とし,$f,g$ の一方が定数,例え
  ば $f=a_0$ なら
  \[
    \resul(a_0, g) = a_0^n = a_0^{n-1}\cdot  f + 0\cdot g
  \]
  とすればよいので,$f,g$ 共に定数でないとし,$\resul(f,g) \neq 0$ とする.まず,
  \[
    \tilde{U} f + \tilde{V}g = 1
  \]
  を満たす多項式 $\tilde{U}, \tilde{V}$
  を構成する.$\ds \tilde{U}=\sum_{i=0}^{n-1} u_i x^i, \;
  \tilde{V}=\sum_{j=0}^{m-1}v_j x^j$ とおくと,
  \[
    \tilde{U}f+\tilde{V}g=1 \Leftrightarrow \left[
      \begin{array}{ccccc}
        a_3 & 0 & b_2 & 0 & 0\\
        a_2 & a_3 & b_a & b_2 & 0\\
        a_1 & a_2 & b_0 & b_1 & b_2\\
        a_0 & a_1 & 0 & b_0 & b_1\\
        0 & a_0 & 0 & 0 & b_0
      \end{array}
    \right] \left[
      \begin{array}{c}
        u_1\\
        u_0\\
        v_2\\
        v_1\\
        v_0
      \end{array}
    \right] = \left[
      \begin{array}{c}
        0\\
        0\\
        0\\
        0\\
        1
      \end{array}
    \right] 
  \]
  である($m=3, n=2$ の場合).この係数行列は $\Syl(f,g)^{\top}$ であり,
  その行列式は $\resul(f,g)\neq 0$ なのでこれは唯一つの解を持つ.クラー
  メルの公式から,例えば $u_1$ は
  \[
    u_1 = \frac{1}{\resul(f,g)} \left|
      \begin{array}{ccccc}
        0 & 0 & b_2 & 0 & 0\\
        0 & a_3 & b_a & b_2 & 0\\
        0 & a_2 & b_0 & b_1 & b_2\\
        0 & a_1 & 0 & b_0 & b_1\\
        1 & a_0 & 0 & 0 & b_0
      \end{array}
    \right|
  \]
  であり,この行列式部分は $R$ の元である.他
  の $u_i, v_j$
  についても同様なので,共通の分母 $\resul(f,g)$ を払って $U = \resul(f,g) \tilde{U}, \; V=
  \resul(f,g) \tilde{V}$ とすれば,$U f + V g = \resul(f,g)$ である.
\end{proof}

\begin{remark}
  $\resul(f,g) \neq 0$ は $\gcd(f,g)=1$
  と同値なので,$\tilde{U} f + \tilde{V} g =1$ となる多項
  式 $\tilde{U},\tilde{V}$ の存在は拡張ユークリッドの互除法により保証さ
  れる.上の定理は,$\tilde{U},\tilde{V}$ の共通の分母が $\resul(f,g)$
  であることを表している.このことから,\cite{CLO}では終結式
  を ``denominator-free'' version of the gcd とも呼んでいる.
\end{remark}



\section{2変数多項式の終結式}


\begin{thebibliography}{99}
\bibitem{NI} 長坂工作・岩根秀直(編),『計算機代数の基礎理論』,共立出版 (2020).

\bibitem{miyake} 三宅敏恒,『線形代数概論』,培風館 (2023).

\bibitem{yokoyama} 横山和弘,『多項式と計算機代数』,朝倉書店 (2022).

\bibitem{CLO} D. A. Cox, J. Little and D. O'Shea, \textit{Ideals Varieties, and Ulgorithms 4th edition}, Springer (2015).

\bibitem{Lang} S. Lang, \textit{Algebra Revised 3rd edition}, Springer (2004). 
\end{thebibliography}

\end{document}
